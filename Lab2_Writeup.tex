\documentclass[titlepage]{scrartcl}
\usepackage{enumitem}
\usepackage[british]{babel}
\usepackage[style=apa, backend=biber]{biblatex}
\DeclareLanguageMapping{british}{british-apa}
\usepackage{url}
\usepackage{float}
\usepackage[labelformat=empty]{caption}
\restylefloat{table}
\usepackage{perpage}
\MakePerPage{footnote}
\usepackage{abstract}
\usepackage{graphicx}
% Create hyperlinks in bibliography
\usepackage{hyperref}
\usepackage{amsmath}

\usepackage[T1]{fontenc}
\usepackage[utf8]{inputenc}
\usepackage{blindtext}
\setkomafont{disposition}{\normalfont\bfseries}

\graphicspath{
    {./resources/},
}
\addbibresource{ ~/Documents/library.bib }

\newsavebox{\abstractbox}
\renewenvironment{abstract}
  {\begin{lrbox}{0}\begin{minipage}{\textwidth}
   \begin{center}\normalfont\sectfont\abstractname\end{center}\quotation}
  {\endquotation\end{minipage}\end{lrbox}%
   \global\setbox\abstractbox=\box0 }

\usepackage{etoolbox}
\makeatletter
\expandafter\patchcmd\csname\string\maketitle\endcsname
  {\vskip\z@\@plus3fill}
  {\vskip\z@\@plus2fill\box\abstractbox\vskip\z@\@plus1fill}
  {}{}
\makeatother

\DeclareCiteCommand{\citeyearpar}
    {}
    {\mkbibparens{\bibhyperref{\printdate}}}
    {\multicitedelim}
    {}

\newcommand{\code}[1]{\texttt{#1}}

% MATLAB Code block stuff...
\usepackage{color}
\usepackage{listings}

\definecolor{dkgreen}{rgb}{0,0.6,0}
\definecolor{gray}{rgb}{0.5,0.5,0.5}

\lstset{language=Matlab,
    keywords={break,case,catch,continue,else,elseif,end,for,function,
        global,if,otherwise,persistent,return,switch,try,while},
    basicstyle=\ttfamily,
    keywordstyle=\color{blue},
    commentstyle=\color{gray},
    stringstyle=\color{dkgreen},
    numbers=left,
    numberstyle=\tiny\color{gray},
    stepnumber=1,
    numbersep=10pt,
    backgroundcolor=\color{white},
    tabsize=4,
    showspaces=false,
    showstringspaces=false
    frame=single,
    breaklines=true,
    %postbreak=\raisebox{0ex}[0ex][0ex]{\ensuremath{\color{red}\hookrightarrow\space}}
}
\begin{document}
\title{ECS707P - DSP}
\subtitle{\LARGE{DSP Lab 2 Writeup}}
\author{Sam Perry - EC16039}

\maketitle

\section{Generating $x[n] = sin(2\pi128n /2048)$}
\begin{lstlisting}
% Main function for running exercises
function main()
    % Declare the number of samples in the sine wave
    N = 32;
    % Declare the sample rate
    fs = 2048;
    % Create indexes 1 - N
    time = [1:N];
    % Declare the frequency as 128Hz
    frequency = 128;
    % Declare the amplitude
    amplitude = 1.0;

    % Generate signal x from parameters
    x = amplitude * sin(2*pi*frequency*(time/2048));

    % plot x to lollipop graph
    stem(x)
\end{lstlisting}

\begin{figure}[H]
    \caption{\code{stem} Output Plot}
    \makebox[\textwidth]{\includegraphics[width=\textwidth]{SineStem}}
\end{figure}

\section{Generating DFT using \code{fft}}

\begin{lstlisting}
% Main function for running exercises
function main()
    % Declare the number of samples in the sine wave
    N = 32;
    % Declare the sample rate
    fs = 2048;
    % Create indexes 1 - N
    time = [0:N-1];
    % Declare the frequency as 128Hz
    frequency = 128;
    % Declare the amplitude
    amplitude = 1.0;

    % Generate signal x from parameters
    x = amplitude * sin(2*pi*frequency*(time/2048));

    % Compute the DFT of the sine wave
    X = fft(x);

    % Create a new figure window
    figure
    % Take only the values < the nyquist frequency (1024)
    X = abs(X)'
    X = X(1:N/2)
    % Plot the absolute magnitude of the DFT to a graph
    stem(X);

    % Turn on grid
    grid on;
    
    % Add a title to the graph
    title('DFT of x')
    
    % Create labels for x ticks
    labels = 0:(fs/2)/(N/2):fs/2
    % Setup x ticks
    set(gca, 'XTick', 0:length(labels)); % Change x-axis ticks
    set(gca, 'XTickLabel', labels); % Change x-axis ticks labels.
    % Label the x axis as 'Frequency'
    xlabel('Frequency (Hz)')
    % Label the y axis as 'Magnitude'
    ylabel('Magnitude |X|')
\end{lstlisting}

\section{What is the phase at the frequency 128 Hz?}
The angle returned is -1.5708 ($\frac{-\pi}{2}$). This is because the
difference between the sinusoid generated and the cosine of the same frequency
at this bin is $\frac{-\pi}{2}$.

\section{sample a sinusoid at 220 Hz at the same sampling rate}
\begin{figure}[H]
    \caption{\code{stem} Plot of a Sine Wave at 220Hz}
    \makebox[\textwidth]{\includegraphics[width=\textwidth]{Sine220Stem}}
\end{figure}

\section{Draw a picture of what you hypothesize will be the magnitude DFT of
this signal}
It is expected that two frequency bins closest will have magnitudes $> 0$. This
is because there is not frequency bin that represents 220Hz exactly and the
bins either side will correlate a certain amount with the sinusoid.

\section{Find the DFT of this signal and plot the magnitude of the result versus frequency}
\begin{figure}[H]
    \caption{32 Bin DFT of a Sine Wave at 220Hz}
    \makebox[\textwidth]{\includegraphics[width=\textwidth]{DFT220}}
\end{figure}

\subsection{How does this plot compare to that in 1.2?}
There is significant spectral content across all frequency bins, with a peak
at bins around the frequency of the sinusoid generated. This contrasts the
single peak of magnitude $\frac{N}{2}$ at the exact frequency of the sinusoid
generated in section 1.2.
\subsection{Was your hypothesized plot, created in 1.5, correct?}
The hypothesized plot drawn in section 1.5 was partially correct. It predicted
the dispersion of magnitude across the nearest frequency bins due to the lack
of an exact bin for that frequency, however it did not anticipate the
distribution of magnitudes across all frequency bins due to the discontinuity
in signal as stated in the lab document.

\section{Create a 512-length sequence of the sinusoid in equation (3) with a
frequency of 220 Hz, amplitude of 1, and a sampling rate of 2,048 Hz}
\begin{figure}[H]
    \caption{\code{512 Bin DFT of a Sine Wave at 220Hz}}
    \makebox[\textwidth]{\includegraphics[width=\textwidth]{DFT220_512}}
\end{figure}

\section{Comment the program, line-by- line, to describe what it is doing}
\begin{lstlisting}
% Main function for running exercises
function main()
    % Clear all previously defined variables
    clear all
    % Initialize string with name of sound file
    sndfile = 'speech_female.wav';
    % Read the audio samples of the sound file into matrix x and save the
    % integer samplerate in variable Fs
    [x,Fs] = audioread(sndfile);
    % Set the variable N to 512
    N = 512;

    % Returns a matrix S, Frequency range F and Timestemp T for 1.4second of
    % the input vector x. Argument 2 is the window size used for windows.
    % Argument 3 is the fft size for each window. Larger values will results in
    % a higher frequency resolution, lower values results in a better temporal
    % resolution. Argument 4 is the sample rate of the signal x.
    [S,F,T] = spectrogram(x(1:Fs*1.4),N,3*N/4,N*4,Fs);
    % Generate a figure with set position and other display related parameters.
    f = figure('Position',[500 300 700 500],'MenuBar','none', 'Units','Normalized');
    % More display related settings...
    set(f,'PaperPosition',[0.25 1.5 8 5]);
    % same as above...
    axes('FontSize',14);
    % same as above...
    colormap('jet');

    % Display spectrogram over time T, with frequency in KHz. Matrix of values
    % are converted to absolute magnitudes and scaled to decibels.
    imagesc(T,F./1000,20*log10(abs(S)));
    % Set labels of spectrogram image
    axis xy; set(gca,'YTick',[0:2000:Fs/2]./1000,'YTickLabel',[0:2000:Fs/2]./1000);
    ylabel('Frequency (kHz)'); xlabel('Time (s)');
    print(gcf,'-depsc2','p2i1.eps');
\end{lstlisting}

\section{Subsection}
\subsection{In what frequency band is most of the energy located?}
For the majority of the sample, frequency of 5-6KHz and below are most
prominent. The exception is during noisy phonemes where the energy is spread
across higher frequencies.
\subsection{Knowing the Nyquist-Shannon Sampling Theorem, at what minimum rate
    would you sample this signal such that it could be reconstructed well—but
not exactly?}
From visual inspection, it may be possible to reconstruct the sound to an
intelligible degree at a sample rate of around 11KHz. It may be possible to
lower this further.

\section{Which sounds of which words correspond to the energy between 4 and 18
kHz?} These frequency ranges contain the most energy during sibilant phonemes.
This is clearly demonstrated in the word ``administer'' at the `s'.

\section{What is the highest decimation factor you think this signal can
    withstand and you can still understand the speech? What would the sampling
rate be at that factor?}
A decimation factor 6 was found to be the highest factor that was still
intelligible. This results in a sample rate of 7350Hz.

\section{Do it. Include your code. Include your observations.}
\begin{lstlisting}
% Main function for running exercises
function main()
    % Clear all previously defined variables
    clear all
    % Initialize string with name of sound file
    sndfile = 'speech_female.wav';
    % Read the audio samples of the sound file into matrix x and save the
    % integer samplerate in variable Fs
    [x,Fs] = audioread(sndfile);
    % Set the variable N to 512
    
    % Set decimation factor
    decimation_factor = 16;
    % Pick every $decimation_factor samples
    x = x(1:decimation_factor:length(x));
    % Set sample rate accordingly
    Fs = Fs / decimation_factor;
    % Play sound.
    sound(x(1:Fs*1.4), Fs);
    return;
\end{lstlisting}

Decimation is applied by simply taking every nth sample. A more sophisticated
approach would be to apply a low pass filter to the audio before decimation in
order to avoid aliasing.

\section{Say you have an animal that needs medicine. Knowing what you know now,
how would you administer said medicine?}
Although it is a very difficult matter, the simplest method is to mix the
medicine with butter, or some other grease, and smear it on the nose of the
animal from time to time...
\end{document}
