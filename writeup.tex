\documentclass[titlepage]{scrartcl}
\usepackage{enumitem}
\usepackage[british]{babel}
\usepackage[style=apa, backend=biber]{biblatex}
\DeclareLanguageMapping{british}{british-apa}
\usepackage{url}
\usepackage{float}
\usepackage[labelformat=empty]{caption}
\restylefloat{table}
\usepackage{perpage}
\MakePerPage{footnote}
\usepackage{abstract}
\usepackage{graphicx}
% Create hyperlinks in bibliography
\usepackage{hyperref}
\usepackage{amsmath}

\usepackage[T1]{fontenc}
\usepackage[utf8]{inputenc}
\usepackage{blindtext}
\setkomafont{disposition}{\normalfont\bfseries}

\graphicspath{
    {./resources/},
}
\addbibresource{~/Documents/library.bib}

\newsavebox{\abstractbox}
\renewenvironment{abstract}
  {\begin{lrbox}{0}\begin{minipage}{\textwidth}
   \begin{center}\normalfont\sectfont\abstractname\end{center}\quotation}
  {\endquotation\end{minipage}\end{lrbox}%
   \global\setbox\abstractbox=\box0 }

\usepackage{etoolbox}
\makeatletter
\expandafter\patchcmd\csname\string\maketitle\endcsname
  {\vskip\z@\@plus3fill}
  {\vskip\z@\@plus2fill\box\abstractbox\vskip\z@\@plus1fill}
  {}{}
\makeatother

\DeclareCiteCommand{\citeyearpar}
    {}
    {\mkbibparens{\bibhyperref{\printdate}}}
    {\multicitedelim}
    {}

% MATLAB Code block stuff...
\usepackage{color}
\usepackage{listings}

\definecolor{dkgreen}{rgb}{0,0.6,0}
\definecolor{gray}{rgb}{0.5,0.5,0.5}

\lstset{language=Matlab,
   keywords={break,case,catch,continue,else,elseif,end,for,function,
      global,if,otherwise,persistent,return,switch,try,while},
   basicstyle=\ttfamily,
   keywordstyle=\color{blue},
   commentstyle=\color{gray},
   stringstyle=\color{dkgreen},
   numbers=left,
   numberstyle=\tiny\color{gray},
   stepnumber=1,
   numbersep=10pt,
   backgroundcolor=\color{white},
   tabsize=4,
   showspaces=false,
   showstringspaces=false}

\begin{document}
    \title{ECS707P Fundamentals of DSP}
    \subtitle{\LARGE{Lab 1 Report}}
    \author{Sam Perry - ec16039}

    \maketitle

    \section*{Section 2.1 Buoy data parsing function}
    \begin{lstlisting}
    % Declare a function that takes a signle argument (the
    % file path) and returns a 'data' object containing the
    % data parsed from the file object.  A count integer is
    % also returned with the number of entries read from
    % file.
    function [data, count] = readbuoydata(datafile)

        % Create a file object with the path provided by the
        % 'datafile' variable
        fid = fopen(datafile,'r');
        % Read first two header lines of file so that they
        % are ignored in processing below.
        tline = fgetl(fid);
        tline = fgetl(fid);

        % Parse each line of the file from line 3 onwards
        % and store in variable 'A' first argument specifies
        % the file object.
        % second arguments specifies the data type to expect
        % for each of the 10 elements per line.
        % the third argument specifies to read 10 elements
        % per line, and to read until the end of the file.
        [A,count] = fscanf(fid, ...
        '%d %d %d %d %d %f %f %d %f %f',[10 inf]);

        % Creates a member of the data object that contains
        % dates in the 'serial date number' format
        data.date = datenum([A(1:5,:); zeros(1,size(A,2))]')';
        data.Hs = A(6,:); % significant wave height
        data.Tp = A(7,:); % peak period
        data.Dp = A(8,:); % peak period direction
        data.Ta = A(9,:); % average period
        data.SST = A(10,:); % sea surface temperature

        % Close the file 
        fclose(fid);
    \end{lstlisting}

    \section*{Section 2.2 Peak Period and Wave Height Plots}

    \begin{figure}[H]
        \makebox[\textwidth]{\includegraphics[width=1.3\textwidth]{HeightPeriodPlot}}
    \end{figure}

    \section*{Section 2.3 Moving Average Filter Function}
    \begin{lstlisting}
    % Function takes an 1-dimensional input signal and
    % an M value (>0) 
    % A filtered signal of the same size is returned
    function [outputSignal] = movingAverage(inputSignal, M)
        % Initialize output array with zeros.
        outputSignal = zeros(1, length(inputSignal));
        % Pad input with zeros at the begining.
        inputSignal = [zeros(1, M-1), inputSignal];

        % For each sample in input...
        for n = M:length(inputSignal)
            % Take the last M samples and save their
            % mean value as the output sample.
            outputSignal(n-M+1) = mean(inputSignal(n-M+1:n));
        end
    \end{lstlisting}

    \section*{Section 2.4 Filtered Peak Period Data Plots}
    FIX Plot to show "(s)"
    \begin{figure}[H]
        \makebox[\textwidth]{\includegraphics[width=1.3\textwidth]{MovingAveragePlot}}
    \end{figure}

    \section*{Section 2.5 Questions}
    \subsection*{Q1. What do you observe as $M$ increases?}
    The output signal appears to be smoothed at increasing
    amounts as values for $M$ increase.
    \subsection*{Q2. Why do you think you observe this ``thing''?}
    Increasing $M$ increases the size of the ``slice'' of
    signal averaged to form each samples in the output
    signal. By averaging larger slices of signal to form the
    output, output samples will deviate less from the
    previous sample due to the increased similarity between
    the consecutive slices. This causes the smoothing effect
    observed.
    \subsection*{Q3. What is happening at the beginning of the averaged dataset, and why does it happen?}

    \subsection*{Q4. What happens to the running average when the peak period suddenly drops?}
    \subsection*{Q5. Are these drops preserved?}
    \subsection*{Q6. Are the wave trains more clear?}

    \section*{Section 2.6 Filtered Wave Height Plots}
    \begin{figure}[H]
        \makebox[\textwidth]{\includegraphics[width=1.3\textwidth]{MovingAverageWavePlot}}
    \end{figure}

    \section*{Section 2.7 Questions}
    \subsection*{Q1. Do you observe anything different in the plots from 2.6 to those you saw in 2.4?}
    \subsection*{Q2. How does the size of this shift relate to M?}
    \subsection*{Q3. Explain why the peaks move?}

    % \printbibliography
\end{document}
